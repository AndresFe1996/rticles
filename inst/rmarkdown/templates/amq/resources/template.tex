%%%%%%%%%%%%%%%%%%%%%%%%%%%%%%%%%%%%%%%%%%%%%%%%%
%%%%%%%%%%%%%%%%%%%%%%%%%%%%%%%%%%%%%%%%%%%%%%%%%
%%
%%  Modèle pour le  Bulletin de l'Association
%%           mathématique du Québec
%%
%%                 Décembre 2017
%%       Adapté par Marc-André Désautels
%%
%%%%%%%%%%%%%%%%%%%%%%%%%%%%%%%%%%%%%%%%%%%%%%%%%
%%%%%%%%%%%%%%%%%%%%%%%%%%%%%%%%%%%%%%%%%%%%%%%%%


\documentclass[10pt]{article}

%=====================================Ensemble des packages================================

% Ce package sert à accueillir les accents et à faire
% les césures comme on aime qu'elles soient faites en français.
\usepackage[T1]{fontenc}

% Choix de la fonte
\usepackage{lmodern}

%% Permet de traiter les textes accentués
% Pour Windows
\usepackage[utf8]{inputenc}
% Pour Unix
%\usepackage[latin1]{inputenc}
% Pour Apple
%\usepackage[applemac]{inputenc}

% Ensuite, entrer le texte accentué comme à l'ordinaire.
\usepackage[english,french]{babel}

% Permet de rendre le fichier maître plus compatible,
% en particulier avec l'insertion de figures,
% (sauf celle de figures flottantes (package floatingfigure))
% et avec la définition des caractères à doubles jambes.
\usepackage{graphicx}

% Packages qui permettent l'utilisation des symboles et des commandes
% mathématiques de AMS-LaTeX,
\usepackage{amssymb,latexsym,amsmath}

% Package qui donne accès à une police calligraphique, comme eucal.
\usepackage{mathrsfs}

% Package qui permet de fusionner des lignes ou des colonnes de tableaux
\usepackage{multirow}

% Package qui donne accès à la fonte Euler.
\usepackage{euscript}

% Package qui permet de définir ses propres en-têtes et pieds de page.
\usepackage{fancyhdr}

% Package qui permet de passer du texte sur le fichier
% pdf à la ligne du fichier source correspondant.
\usepackage{pdfsync}

% Package permettant de définir son propre style
% de présentation des théorèmes.
\usepackage{theorem}

% Package qui permet au texte de s'enrouler autour de flotants.
\usepackage{wrapfig}
\usepackage{floatflt}

% Package qui "rectifie" les caractères qui dépassent dans la marge.
\usepackage[protrusion=true]{microtype}

% Permet de tracer des graphes flèchés.
% Voir http://www.tug.org/TUGboat/tb22-4/tb72perlax.pdf
\usepackage[all]{xy}

%  Permet de définir les dimensions de la partie imprimée de la page.
% Voir http://ctan.cms.math.ca/tex-archive/macros/latex/contrib/geometry/geometry.pdf
\usepackage{geometry}

% Permet, entre autres, de mettre du texte en filigrane.
% Voir http://math.et.info.free.fr/TikZ/bdd/TikZ-Impatient.pdf
% p. 165
\usepackage{tikz}

% Permet d'utiliser le gabarit de tableaux utilisant les commandes
% of \toprule, \midrule and \bottomrule pour les lignes horizontales dans les tableaux.
\usepackage{booktabs}

\usepackage{color}

% Permet la présentation d'adresses url sous leur forme habituelle.
% Voir http://www.tex.ac.uk/cgi-bin/texfaq2html?label=setURL
\usepackage{url}

% Permet de rendre actifs les liens créés avec le package précédent.
% Voir /usr/local/share/texmf-dist/doc/latex/hyperref/manual.pdf
\usepackage[colorlinks=true,urlcolor=blue, citecolor=cyan, linkcolor=magenta]{hyperref}

% Pour une adresse de courriel: \href{mailto:foo@bar.abc}{Good Link!}
% Pour de belles flèches sur les vecteurs, cadeau de Jean-Philippe Morin
\usepackage{esvect}

%==================================Fin de l'ensemble des packages============================


%=================Pour de belles flèches sur les vecteurs, cadeau de Jean-Philippe Morin=====
\renewcommand{\vec}[1]{\protect\vv{#1}}
% notation de vecteur (\vv définie par package esvect)
% Exemple: \vv{v} ou \vv{AB} ou \vv{A_1A_2}
%============================================================================================


%%%======= Début macro pour numérotation des équations selon les sections du texte==========
%\makeatletter
%\renewcommand\theequation{\thesection.\arabic{equation}}
%\@addtoreset{equation}{section}
%\makeatother
%%%======================  macro: mode d'emploi ============================================
%%% Pour activer cette macro commande, enlever le symbole % au début de la ligne suivante
%\numberwithin{equation}{section}
%%============================================================================================




%%===========Les énoncés ayant la même présentation que les théorèmes=========================
\newtheorem{theorem}{Théorème}
%[section]
\newtheorem{lemma}{Lemme}
%[section]
\newtheorem{proposition}{Proposition}
%[section]
\newtheorem{corollary}{Corollaire}
%[section]
\newtheorem{definition}{Définition}
%[section]
\newtheorem{nota}{Notation}
%[section]
\newtheorem{ex}{Exemple}
%[section]
\newtheorem{nb}{N.B.}
%[section]
\newtheorem{remark}{Remarque}
%%============================================================================================


%%==========================Les démonstrations============================
\newenvironment{proof}
{\par\noindent
\textit{Démonstration}\ }
{
\hfill{$$\Box$$}
}

%%==============================Les siècles====================================================
%%        Pour la notation des siècles en chiffres romains
%%        La commande est  \siecle{#}
%%        Exemple  \siecle{21}.
%%==========================================================================================
\newcommand{\siecle}[1]{%
\ifnum #1=1%
\uppercase\expandafter{\romannumeral #1}\textsuperscript{er}%
\else%
\uppercase\expandafter{\romannumeral #1}\up{e}%
\fi}

%%=====================================================================
%   Définition des marges pour le Bulletin (NE PAS MODIFIER)
\geometry{height=18.6cm,width=14.4cm}
\geometry{hmargin={3.6cm,3.6cm}}
\geometry{vmargin={4.3cm,5cm}}
%%Ces trois lignes doivent rester avant le \begin{document}.
\parindent=0.0in
\parskip=0.1in
\setcounter{page}{1}
%%=====================================================================

%%========================Caractères particuliers usuels======================
%%
%%==============================Double jambe.============================
\DeclareSymbolFont{AMSb}{U}{msb}{m}{n}
\DeclareSymbolFontAlphabet{\Bbb}{AMSb}
\newcommand{\ds}{\displaystyle}

\usepackage[]{lineno}

\newcommand*\patchAmsMathEnvironmentForLineno[1]{%
  \expandafter\let\csname old#1\expandafter\endcsname\csname #1\endcsname
  \expandafter\let\csname oldend#1\expandafter\endcsname\csname end#1\endcsname
  \renewenvironment{#1}%
     {\linenomath\csname old#1\endcsname}%
     {\csname oldend#1\endcsname\endlinenomath}}%
\newcommand*\patchBothAmsMathEnvironmentsForLineno[1]{%
  \patchAmsMathEnvironmentForLineno{#1}%
  \patchAmsMathEnvironmentForLineno{#1*}}%
\AtBeginDocument{%
\patchBothAmsMathEnvironmentsForLineno{equation}%
\patchBothAmsMathEnvironmentsForLineno{align}%
\patchBothAmsMathEnvironmentsForLineno{flalign}%
\patchBothAmsMathEnvironmentsForLineno{alignat}%
\patchBothAmsMathEnvironmentsForLineno{gather}%
\patchBothAmsMathEnvironmentsForLineno{multline}%
}

%%==========================Packages propres à l'auteur==============================
%%
%% Vous pouvez ajouter ici tous les packages que vous avez besoin.
%% Par exemple

% \usepackage{pstricks}
% \usepackage{enumitem}

%%
%%==========================Fin des packages propres à l'auteur=======================

%%==========================Commandes propres à l'auteur==============================
%%
%% Vous pouvez ajouter ici toutes les commandes que vous avez définies.
%% Par exemple

% \newcommand{\lr}[1]{\left(#1\right)}
% \newcommand{\abs}[1]{\left\vert#1\right\vert}

%%
%%==========================Fin des commandes propres à l'auteur========================

\usepackage{url}
\usepackage{hyperref}

%%============================pour des barres de fraction obliques=======================
%running fraction with slash - requires math mode.
\newcommand*\rfrac[2]{{}^{#1}\!/_{#2}}
%	Exemple :	\rfrac{3}{7}
%%===============================================================================


%%=== Pour des barres de fraction latérales, une petite, une grande et une très grande ===
\newcommand{\fracinline}[2]{\raisebox{0.4ex}{$$#1$$} / \raisebox{-0.7ex}{$$#2$$}}
\newcommand{\bigfracinline}[2]{\raisebox{0.8ex}{$$#1$$}  \Big/ \raisebox{-1.4ex}{$$#2$$}}
\newcommand{\biggfracinline}[2]{\raisebox{0.8ex}{$$#1$$}  \Bigg/ \raisebox{-1.4ex}{$$#2$$}}

%%	Exemple:	\fracinline{1 + \cos x}{\sin x}
%%	Exemple:	\bigfracinline{1 + \cos x}{\sin x}
%%	Exemple:	\biggfracinline{1 + \cos x}{\sin x}
%%=======================================================================================

\everymath{\displaystyle}

% Pandoc citation processing
$if(csl-refs)$
\newlength{\csllabelwidth}
\setlength{\csllabelwidth}{3em}
\newlength{\cslhangindent}
\setlength{\cslhangindent}{1.5em}
% for Pandoc 2.8 to 2.10.1
\newenvironment{cslreferences}%
  {$if(csl-hanging-indent)$\setlength{\parindent}{0pt}%
  \everypar{\setlength{\hangindent}{\cslhangindent}}\ignorespaces$endif$}%
  {\par}
% For Pandoc 2.11+
\newenvironment{CSLReferences}[2] % #1 hanging-ident, #2 entry spacing
 {% don't indent paragraphs
  \setlength{\parindent}{0pt}
  % turn on hanging indent if param 1 is 1
  \ifodd #1 \everypar{\setlength{\hangindent}{\cslhangindent}}\ignorespaces\fi
  % set entry spacing
  \ifnum #2 > 0
  \setlength{\parskip}{#2\baselineskip}
  \fi
 }%
 {}
\usepackage{calc} % for calculating minipage widths
\newcommand{\CSLBlock}[1]{#1\hfill\break}
\newcommand{\CSLLeftMargin}[1]{\parbox[t]{\csllabelwidth}{#1}}
\newcommand{\CSLRightInline}[1]{\parbox[t]{\linewidth - \csllabelwidth}{#1}\break}
\newcommand{\CSLIndent}[1]{\hspace{\cslhangindent}#1}
$endif$

$for(header-includes)$
$header-includes$
$endfor$

\begin{document}

%%==========================Éléments propres à l'auteur==============================
%%
%%
%%==========================Fin des éléments propres à l'auteur======================


%%=====================Pour mettre le titre des tableaux en français==================
%%
\renewcommand{\tablename}{\textsc{Tableau}}
%%
%%=========================Fin de la nouvelle définition==============================

\begin{center}
\textsf{\LARGE \textbf{$title$}}
\end{center}

\begin{flushright}
\textsc{$author$, $affiliation$} \\
\textsc{$establishment$} \\
$email$ \\
$website$
\end{flushright}

$if(author2)$
\begin{flushright}
\textsc{$author2$, $affiliation2$} \\
\textsc{$establishment2$} \\
$email2$ \\
$website2$
\end{flushright}
$endif$

\baselineskip=1.2\baselineskip

%%=====================================================================
%%          Résumé
%%=====================================================================

\begin{abstract}

$abstract$

\end{abstract}
%%=====================================================================
%%          Fin du résumé
%%=====================================================================

\textbf{Mots clés: } $keywords$

\bigskip

$body$

\bigskip

\bibliographystyle{alpha}
\begin{thebibliography}{10}

\bibitem{cout} Couture, M. (2010, mise à jour 18 avril). {\em Revue internationale des technologies en édagogie universitaire. Normes bibliographiques - Adaptation française des normes de l'APA}. Récupéré le 3 janvier 2011 du site de l'auteur :  \href {http://www.teluq.uqam.ca/~mcouture/apa}{http://www.teluq.uqam.ca/~mcouture/apa}

\bibitem{daha} Dahan-Dalmedico, A.  et Peiffer, J. (1982). {\em Histoire des mathématiques-Routes et dédales}. Paris, France: Études vivantes.

\bibitem{desg} Desgraupes, B. (2003). {\em \LaTeX, Apprentissage, guide et référence}. France: Vuibert.

\bibitem{ctan} Fairbairns, R. (2006, mise à jour le 21 janvier 2006). {\em Search CTAN For a File}, Récupéré le 3 janvier 2011 du site de CTAN :\href{http://www.tex.ac.uk/CTANfind.html}{http://www.tex.ac.uk/CTANfind.html}.\\
Demander texlive pour arriver à \\tex-archive/systems/texlive/Images/Images/texlive-inst.iso.bz2

Cliquer pour récupérer l'image compressée du CD au format bzip2 (550 Mo!)

\bibitem{flipo} Flipo, D. (2004).  \LaTeX, logiciel libre et gratuit, comme alternative à MS-Word, {\em Quadrature}, no. 54, 12--15.

\bibitem{faq}  Kluth, M.-P. et Bayart, B. (1997, mise à jour 16 octobre 2001). {\em FAQ  \LaTeX~de l'équipe Grappa}. Récupéré le 3 janvier 2011 du site du groupe Grappa de l'Université de Lille 3 : \href{http://www.grappa.univ-lille3.fr/FAQ-LaTeX/}{http://www.grappa.univ-lille3.fr/FAQ-LaTeX/}.

\bibitem{lamp} Lamport, L. (1986). {\em \LaTeX: A Document Preparation System}. Reading, MA: Addison-Wesley Publishing Company.

\bibitem{mass} Masson, B. (2009). \begin{itshape}{\LaTeX, créer ses commandes}\end{itshape}. Récupéré le 3 janvier 2011 du site de l'auteur :\href {http://bertrandmasson.free.fr/wp-content/uploads/commande.pdf}{http://bertrandmasson.free.fr/wp-content/uploads/commande.pdf}.

\bibitem{oeti} Oetiker, T., Partl, H., Hyna, I. et Schlegl, E. (1999). {\em Une courte (?) introduction à \LaTeXe, ou  \LaTeXe~en 88 minutes}, Version 3.3. Récupéré le 3 janvier 2011 du site du Loria : \href{http://tex.loria.fr/general/flshort-3.3.pdf}{http://tex.loria.fr/general/flshort-3.3.pdf}.

\bibitem{roll}Rolland, C. (2000). {\em \LaTeX~par la pratique} ($2^e$ Éd.). Paris, France: Eyrolles.

\end{thebibliography}


\end{document}
